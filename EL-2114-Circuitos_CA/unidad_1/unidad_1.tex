\documentclass[../main.tex]{subfiles}
\graphicspath{{\subfix{./assets}}}
\begin{document}
\everymath{\displaystyle}
\renewcommand{\arraystretch}{1.4}
%-------------------------------
\hrule
\vspace*{-\baselineskip}
\vspace{0.05in}
\section{Unidad 1: Análisis en estado sinusoidal permanente}
\hrule
\begin{xltabular}{\textwidth}{|X|X|}
	\toprule
	\textbf{\large Ondas sinusoidales o senoides}
	$$\bm{v(t)=V_m\sin(\omega t)}$$
	$\begin{aligned}
			 & \bm{V_m} = \text{la amplitud del senoide}                \\
			 & \bm{\omega} = \text{la frecuencia angular en radianes/s} \\
			 & \bm{\omega} t = \text{el argumento del senoide}
	\end{aligned}$
	\newline\newline
	\subfile{assets/graph_v-wt_v-t}
	\subfile{assets/graph_vm-wt}
	\subfile{assets/diag_fasorial}
	&
	\large\textbf{Identidades entre senoides}

	\begin{enumerate}
		\item $\sin(\omega t \pm\angle{180}) = -\sin(\omega t)$
		\item $\cos(\omega t \pm\angle{180})=-\cos(\omega t)$
		\item $\sin(\omega t \pm \angle{90})=\pm \cos(\omega t)$
		\item $\cos(\omega t \pm 90) = \mp = \sin(\omega t)$
	\end{enumerate}

	$$A\cos(\omega t) + B\sin(\omega t) = C \cos(\omega t-\theta)$$
	
	$$C = \sqrt{A^2 +B^2},~~~~\theta =\tan^{-1}\bigg(\frac{B}{A}\bigg)\pm \text{ si } A < 0$$
	
	\subfile{assets/graph_cos-wt_sen-wt}

	\subfile{assets/fasores}

	\subfile{assets/resistor}

	\subfile{assets/inductor}

	\subfile{assets/capacitor}
	\\
	\hline
\end{xltabular}
%----------------------------------------
\end{document}